%% BioMed_Central_Tex_Template_v1.06
%%                                      %
%  bmc_article.tex            ver: 1.06 %
%                                       %

%%IMPORTANT: do not delete the first line of this template
%%It must be present to enable the BMC Submission system to
%%recognise this template!!

%%%%%%%%%%%%%%%%%%%%%%%%%%%%%%%%%%%%%%%%%
%%                                     %%
%%  LaTeX template for BioMed Central  %%
%%     journal article submissions     %%
%%                                     %%
%%          <8 June 2012>              %%
%%                                     %%
%%                                     %%
%%%%%%%%%%%%%%%%%%%%%%%%%%%%%%%%%%%%%%%%%


%%%%%%%%%%%%%%%%%%%%%%%%%%%%%%%%%%%%%%%%%%%%%%%%%%%%%%%%%%%%%%%%%%%%%
%%                                                                 %%
%% For instructions on how to fill out this Tex template           %%
%% document please refer to Readme.html and the instructions for   %%
%% authors page on the biomed central website                      %%
%% http://www.biomedcentral.com/info/authors/                      %%
%%                                                                 %%
%% Please do not use \input{...} to include other tex files.       %%
%% Submit your LaTeX manuscript as one .tex document.              %%
%%                                                                 %%
%% All additional figures and files should be attached             %%
%% separately and not embedded in the \TeX\ document itself.       %%
%%                                                                 %%
%% BioMed Central currently use the MikTex distribution of         %%
%% TeX for Windows) of TeX and LaTeX.  This is available from      %%
%% http://www.miktex.org                                           %%
%%                                                                 %%
%%%%%%%%%%%%%%%%%%%%%%%%%%%%%%%%%%%%%%%%%%%%%%%%%%%%%%%%%%%%%%%%%%%%%

%%% additional documentclass options:
%  [doublespacing]
%  [linenumbers]   - put the line numbers on margins

%%% loading packages, author definitions

\documentclass[twocolumn]{bmcart}% uncomment this for twocolumn layout and comment line below
%\documentclass{bmcart}

%%% Load packages
\usepackage{amsthm,amsmath}
\usepackage{siunitx}
\usepackage{mfirstuc}
%\RequirePackage{natbib}
\usepackage[colorinlistoftodos]{todonotes}
\RequirePackage{hyperref}
\usepackage[utf8]{inputenc} %unicode support
%\usepackage[applemac]{inputenc} %applemac support if unicode package fails
%\usepackage[latin1]{inputenc} %UNIX support if unicode package fails
\usepackage[htt]{hyphenat}

\usepackage{array}
\newcolumntype{L}[1]{>{\raggedright\let\newline\\\arraybackslash\hspace{0pt}}p{#1}}

%%%%%%%%%%%%%%%%%%%%%%%%%%%%%%%%%%%%%%%%%%%%%%%%%
%%                                             %%
%%  If you wish to display your graphics for   %%
%%  your own use using includegraphic or       %%
%%  includegraphics, then comment out the      %%
%%  following two lines of code.               %%
%%  NB: These line *must* be included when     %%
%%  submitting to BMC.                         %%
%%  All figure files must be submitted as      %%
%%  separate graphics through the BMC          %%
%%  submission process, not included in the    %%
%%  submitted article.                         %%
%%                                             %%
%%%%%%%%%%%%%%%%%%%%%%%%%%%%%%%%%%%%%%%%%%%%%%%%%


%\def\includegraphic{}
%\def\includegraphics{}

%%% Put your definitions there:
\startlocaldefs
\endlocaldefs


%%% Begin ...
\begin{document}

%%% Start of article front matter
\begin{frontmatter}

\begin{fmbox}
\dochead{Report from Brainhack}

%%%%%%%%%%%%%%%%%%%%%%%%%%%%%%%%%%%%%%%%%%%%%%
%%                                          %%
%% Enter the title of your article here     %%
%%                                          %%
%%%%%%%%%%%%%%%%%%%%%%%%%%%%%%%%%%%%%%%%%%%%%%

\title{Facilitating Big Data Meta-Analyses for Clinical Neuroimaging through ENIGMA Wrapper Scripts}
\vskip2ex
\projectURL{Project URL: \url{https://github.com/npnl/ENIGMA-Wrapper-Scripts}}

%%%%%%%%%%%%%%%%%%%%%%%%%%%%%%%%%%%%%%%%%%%%%%
%%                                          %%
%% Enter the authors' addresses here        %%
%%                                          %%
%% Repeat \address commands as much as      %%
%% required.                                %%
%%                                          %%
%%%%%%%%%%%%%%%%%%%%%%%%%%%%%%%%%%%%%%%%%%%%%%

\author{[1]Eric Kan, [2]Julia Anglin, [3]Michael Borich, [4]Neda Jahanshad, [4]Paul Thompson, & [5]Sook-Lei Liew*}

\vskip2ex

[1]The Saban Research Institute of Children's Hospital, Los Angeles, California, USA, Department of Pediatrics of the Keck School of Medicine, University of Southern California, Los Angeles, California, USA
[2]Chan Division of Occupational Science and Occupational Therapy, USC
[3]Division of Physical Therapy, Department of Rehabilitation Medicine, Emory University School of Medicine
[4]Imaging Genetics Center, Laboratory of Neuro Imaging, Keck School of Medicine of USC, University of Southern California
[5]Chan Division of Occupational Science and Occupational Therapy, Division of Biokinesiology and Physical Therapy, Department of Neurology of the Keck School of Medicine, USC

*Correspondence: sliew@usc.edu
Neural Plasticity and Neurorehabilitation Laboratory, Chan Division of Occupational Science and Occupational Therapy, Division of Biokinesiology and Physical Therapy, Keck School of Medicine Department of Neurology,
University of Southern California, Los Angeles, 1540 Alcazar Street, CHP 133 MC 9003, 90089, California, USA

%%%%%%%%%%%%%%%%%%%%%%%%%%%%%%%%%%%%%%%%%%%%%%
%%                                          %%
%% Enter short notes here                   %%
%%                                          %%
%% Short notes will be after addresses      %%
%% on first page.                           %%
%%                                          %%
%%%%%%%%%%%%%%%%%%%%%%%%%%%%%%%%%%%%%%%%%%%%%%

\begin{artnotes}
\end{artnotes}

%\end{fmbox}% comment this for two column layout

%%%%%%%%%%%%%%%%%%%%%%%%%%%%%%%%%%%%%%%%%%%%%%
%%                                          %%
%% The Abstract begins here                 %%
%%                                          %%
%% Please refer to the Instructions for     %%
%% authors on http://www.biomedcentral.com  %%
%% and include the section headings         %%
%% accordingly for your article type.       %%
%%                                          %%
%%%%%%%%%%%%%%%%%%%%%%%%%%%%%%%%%%%%%%%%%%%%%%

%\begin{abstractbox}

%\begin{abstract} % abstract
	
%Blank Abstract

%\end{abstract}



%%%%%%%%%%%%%%%%%%%%%%%%%%%%%%%%%%%%%%%%%%%%%%
%%                                          %%
%% The keywords begin here                  %%
%%                                          %%
%% Put each keyword in separate \kwd{}.     %%
%%                                          %%
%%%%%%%%%%%%%%%%%%%%%%%%%%%%%%%%%%%%%%%%%%%%%%

%\vskip1ex

%URL is missing}
%URL is missing}

% MSC classifications codes, if any
%\begin{keyword}[class=AMS]
%\kwd[Primary ]{}
%\kwd{}
%\kwd[; secondary ]{}
%\end{keyword}

%\end{abstractbox}
%
\end{fmbox}% uncomment this for twcolumn layout

\end{frontmatter}

%{\sffamily\bfseries\fontsize{10}{12}\selectfont Project URL: \url{}}

\begin{table*}
\begin{tabular*}{\textwidth}{p{3cm}p{2cm}p{4cm}p{6cm}}
  User Level  & Ease of Use (1-10) 1 = hardest, 10 = easies & Time to implement scripts & Notes \\
    \hline\noalign{\smallskip}
  Novice 1     &     3     &     00:35:25     &     Required walk through support   \\
  Novice 2     &     3     &     00:52:55     &     Required support for basic terminal commands only; then was able to complete independently   \\
  Moderate 1     &     8     &     00:23:45     &     Required no support   \\
  Moderate 2     &     7     &     00:22:10     &     Required no support   \\
  Expert 1     &     8     &     00:11:34     &     Required no support   \\
  Expert 2 - different enviornment     &     8     &     02:00:00     &     Getting scripts to run took several minutes but reorganizing data and troubleshooting with freesurfer took significant time   \\ 
  Expert 3 - different enviornment     &     9     &     01:15:00     &     Required walk through support   \\ 
  \noalign{\smallskip}\hline
\end{tabular*}
\end{table*}

%%% Import the body from pandoc formatted text
\section{Introduction}\label{introduction}

A vast number of clinical disorders may involve changes in brain
structure that are correlated with cognitive function and behavior
(e.g., depression, schizophrenia, stroke, etc.). Reliably understanding
the relationship between specific brain structures and relevant
behaviors in worldwide clinical populations could dramatically improve
healthcare decisions around the world. For instance, if a reliable
relationship between brain structure after stroke and functional motor
ability was established, brain imaging could be used to predict
prognosis/recovery potential for individual patients. However, high
heterogeneity in clinical populations in both individual neuroanatomy
and behavioral outcomes make it difficult to develop accurate models of
these potentially subtle relationships.

Large neuroimaging studies (n\textgreater{}10,000) would provide
unprecedented power to successfully relate clinical neuroanatomy changes
with behavioral measures. While these sample sizes might be difficult
for any one individual to collect, the ENIGMA Center for Worldwide
Medicine, Imaging, and Genomics has successfully pioneered meta- and
mega-analytic methods to accomplish this task. ENIGMA brings together a
global alliance of over 500 international researchers from over 35
countries to pool together neuroimaging data on different disease states
in hopes of discovering critical brain-behavior relationships
(http://enigma.ini.usc.edu; Thompson et al., 2015). Individual
investigators with relevant data run ENIGMA analysis protocols on their
own data and send back an output folder containing the analysis results
to be combined with data from other sites for a meta-analysis. In this
way, large sample sizes can be acquired without the hassle of
large-scale data transfers or actual neuroimaging data sharing.

ENIGMA protocols were initially developed to harmonize processing
methods of imaging researchers around the world and they require a
moderate level of familiarity with several programming languages and
environments. However, ENIGMA's recent success has attracted greater
interest in collaborative neuroimaging and protocols must be adjusted to
allow for all levels of experience, as, the success of this approach
depends on individual collaborators running these ENIGMA protocols on
their data. Here, we worked on simplifying these protocols so even a
novice programmer could use them. In this way, we hope to expand the
feasibility of collecting critical clinical data from collaborators who
may have less experience with neuroimaging techniques.

\section{Approach}\label{approach}

The current ENIGMA protocols \url{(http://enigma.ini.usc.edu/protocols/)} for
structural neuroimaging analyses consist of a number of different word
documents with embedded links to different scripts and snippets of code
that use R, bash scripting, Matlab, FSL, and Freesurfer. Each step must
be run sequentially, costing the user time during the implementation to
wait for each step to finish before beginning the next. In addition, the
number of different scripts, programming languages and software
environments can be challenging for a novice user and introduces
numerous instances where the individual may make errors in implementing
the code. To address this, we created three easy-to-use wrapper scripts
that automate the implementation of the ENIGMA protocols for both
subcortical and cortical structural MRI analyses. These wrapper scripts
reduce over 40 steps down to 3 quick steps for the user. We also created
a user-friendly readme file that includes screenshots of the code
implementation.

To examine the ease of use and time to implement the new scripts, we
tested each of them on 6 users who had different levels of familiarity
with programming and neuroimaging (novice users (no programming
experience), moderate users (basic-to-intermediate programming
experience), and expert users (extensive programming)). To explore
additional factors relating to implementation, two of the expert users
had to use the scripts in a different environment (e.g., organize the
data, install the software, etc.).

\section{Results/Discussion}\label{resultsdiscussion}

Overall, moderate and expert level users found the scripts extremely
easy to implement and required less than 25 minutes to get all three
scripts running (excluding the run time of each script; see Table 1 for
individual results). The two novice users required greater support to
understand basic elements (e.g., what is a terminal), but with support,
were able to complete all the steps in less than 1 hour. Finally, the
expert users who had to reorganize the environment took 2-3 hours, but
about 10 minutes was script implementation and the remainder of the time
was troubleshooting Freesurfer errors.

The wrapper scripts made the implementation of the ENIGMA protocols
quick and feasible even for novice users. However, there were still
three main barriers to participation that required significant time,
computational resources, and some expertise: 1) data organization
(depending on previous data structure), 2) running freesurfer
(\textasciitilde{}12 hrs/subj), and 3) installation of the required
software (e.g., Freesurfer, FSL, R). Future projects may look at ways to
streamline these areas for a more seamless user experience in order to
facilitate greater sharing of clinical neuroimaging data through ENIGMA.

All scripts and user guides can be found at:
\url{https://github.com/npnl/ENIGMA-Wrapper-Scripts}

%%%%%%%%%%%%%%%%%%%%%%%%%%%%%%%%%%%%%%%%%%%%%%
%%                                          %%
%% Backmatter begins here                   %%
%%                                          %%
%%%%%%%%%%%%%%%%%%%%%%%%%%%%%%%%%%%%%%%%%%%%%%

\begin{backmatter}

\section*{Availability of Supporting Data}
More information about this project can be found at: \emph{URL is missing} 

\section*{Competing interests}
\emph{Competing interests statement is missing.}

\section*{Author's contributions}
\emph{Author's contributions statement is missing.}

\section*{Acknowledgements}
\emph{Acknowedgements section is missing.}

  
  
%%%%%%%%%%%%%%%%%%%%%%%%%%%%%%%%%%%%%%%%%%%%%%%%%%%%%%%%%%%%%
%%                  The Bibliography                       %%
%%                                                         %%
%%  Bmc_mathpys.bst  will be used to                       %%
%%  create a .BBL file for submission.                     %%
%%  After submission of the .TEX file,                     %%
%%  you will be prompted to submit your .BBL file.         %%
%%                                                         %%
%%                                                         %%
%%  Note that the displayed Bibliography will not          %%
%%  necessarily be rendered by Latex exactly as specified  %%
%%  in the online Instructions for Authors.                %%
%%                                                         %%
%%%%%%%%%%%%%%%%%%%%%%%%%%%%%%%%%%%%%%%%%%%%%%%%%%%%%%%%%%%%%

% if your bibliography is in bibtex format, use those commands:

\section{References}\label{references}
ENIGMA and the Individual: Predicting Factors that Affect the Brain in 35 Countries Worldwide.Paul M. Thompson, Ole A. Andreassen, Alejandro Arias-Vasquez, Carrie E. Bearden, Premika S. Boedhoe, Rachel M. Brouwer, Randy L. Buckner, Jan K. Buitelaar, Kazima B. Bulaeva, Dara M. Cannon, Ronald A. Cohen, Patricia J. Conrod, Anders M. Dale, Ian J. Deary, Emily L. Dennis, Marcel A. de Reus, Sylvane Desrivieres, Danai Dima, Gary Donohoe, Simon E. Fisher, Jean-Paul Fouche, Clyde Francks, Sophia Frangou, Barbara Franke, Habib Ganjgahi, Hugh Garavan, David C Glahn, Hans J Grabe, Tulio Guadalupe, Boris A Gutman, Ryota Hashimoto, Derrek P Hibar, Dominic Holland, Martine Hoogman, Hilleke E Hulshoff Pol, Norbert Hosten, Neda Jahanshad, Sinead Kelly, Peter Kochunov, William S Kremen, Phil H Lee, Scott Mackey, Nicholas G Martin, Bernard Mazoyer, Colm McDonald, Sarah E Medland, Rajendra A Morey, Thomas E Nichols, Tomas Paus, Zdenka Pausova, Lianne Schmaal, Gunter Schumann, Li Shen, Sanjay M Sisodiya, Dirk JA Smit, Jordan W Smoller, Dan J Stein, Jason L Stein, Roberto Toro, Jessica A Turner, Martijn van den Heuvel, Odile A van den Heuvel, Theo GM van Erp, Daan van Rooij, Dick J Veltman, Henrik Walter, Yalin Wang, Joanna M Wardlaw, Christopher D Whelan, Margaret J Wright, Jieping Ye, for the ENIGMA Consortium.NeuroImage, in press(in press), pp. 1-7. 2015.

Common genetic variants influence human subcortical brain structures. Derrek P. Hibar, Jason L. Stein, Miguel E. Renteria, Alejandro Arias-Vasquez, Sylvane Desrivières, Neda Jahanshad, Roberto Toro, Katharina Wittfeld, Lucija Abramovic, Micael Andersson, Benjamin S. Aribisala, Nicola J. Armstrong, Manon Bernard, Marc M. Bohlken, Marco P. Boks, Janita Bralten, Andrew A. Brown, M. Mallar Chakravarty, Qiang Chen, Christopher R. K. Ching, Gabriel Cuellar-Partida, Anouk den Braber, Sudheer Giddaluru, Aaron L. Goldman, Oliver Grimm, Tulio Guadalupe, Johanna Hass, Girma Woldehawariat, Avram J. Holmes, Martine Hoogman, Deborah Janowitz, Tianye Jia, Sungeun Kim, Marieke Klein, Bernd Kraemer, Phil H. Lee, Loes M. Olde Loohuis, Michelle Luciano, Christine Macare, Karen A. Mather, Manuel Mattheisen, Yuri Milaneschi, Kwangsik Nho, Martina Papmeyer, Adaikalavan Ramasamy, Shannon L. Risacher, Roberto Roiz-Santiañez, Emma J. Rose, Alireza Salami, Philipp G. Sämann, Lianne Schmaal, Andrew J. Schork, Jean Shin, Lachlan T. Strike, Alexander Teumer, Marjolein M. J. van Donkelaar, Kristel R. van Eijk, Raymond K. Walters, Lars T. Westlye, Christopher D. Whelan, Anderson M. Winkler, Marcel P. Zwiers, Saud Alhusaini, Lavinia Athanasiu, Stefan Ehrlich, Marina M. H. Hakobjan, Cecilie B. Hartberg, Unn K. Haukvik, Angelien J. G. A. M. Heister, David Hoehn, Dalia Kasperaviciute, David C. M. Liewald, Lorna M. Lopez, Remco R. R. Makkinje, Mar Matarin, Marlies A. M. Naber, D. Reese McKay, Margaret Needham, Allison C. Nugent, Benno Pütz, Natalie A. Royle, Li Shen, Emma Sprooten, Daniah Trabzuni, Saskia S. L. van der Marel, Kimm J. E. van Hulzen, Esther Walton, Christiane Wolf, Laura Almasy, David Ames, Sampath Arepalli, Amelia A. Assareh, Mark E. Bastin, Henry Brodaty, Kazima B. Bulayeva, Melanie A. Carless, Sven Cichon, Aiden Corvin, Joanne E. Curran, Michael Czisch, Greig I. de Zubicaray, Allissa Dillman, Ravi Duggirala, Thomas D. Dyer, Susanne Erk, Iryna O. Fedko, Luigi Ferrucci, Tatiana M. Foroud, Peter T. Fox, Masaki Fukunaga, J. Raphael Gibbs, Harald H. H. Göring, Robert C. Green, Sebastian Guelfi, Narelle K. Hansell, Catharina A. Hartman, Katrin Hegenscheid, Andreas Heinz, Dena G. Hernandez, Dirk J. Heslenfeld, Pieter J. Hoekstra, Florian Holsboer, Georg Homuth, Jouke-Jan Hottenga, Masashi Ikeda, Clifford R. Jack Jr, Mark Jenkinson, Robert Johnson, Ryota Kanai, Maria Keil, Jack W. Kent Jr, Peter Kochunov, John B. Kwok, Stephen M. Lawrie, Xinmin Liu, Dan L. Longo, Katie L. McMahon, Eva Meisenzahl, Ingrid Melle, Sebastian Mohnke, Grant W. Montgomery, Jeanette C. Mostert, Thomas W. Mühleisen, Michael A. Nalls, Thomas E. Nichols, Lars G. Nilsson, Markus M. Nöthen, Kazutaka Ohi, Rene L. Olvera, Rocio Perez-Iglesias, G. Bruce Pike, Steven G. Potkin, Ivar Reinvang, Simone Reppermund, Marcella Rietschel, Nina Romanczuk-Seiferth, Glenn D. Rosen, Dan Rujescu, Knut Schnell, Peter R. Schofield, Colin Smith, Vidar M. Steen, Jessika E. Sussmann, Anbupalam Thalamuthu, Arthur W. Toga, Bryan J. Traynor, Juan Troncoso, Jessica A. Turner, Maria C. Valdés Hernández, Dennis van ?t Ent, Marcel van der Brug, Nic J. A. van der Wee, Marie-Jose van Tol, Dick J. Veltman, Thomas H. Wassink, Eric Westman, Ronald H. Zielke, Alan B. Zonderman, David G. Ashbrook, Reinmar Hager, Lu Lu, Francis J. McMahon, Derek W. Morris, Robert W. Williams, Han G. Brunner, Randy L. Buckner, Jan K. Buitelaar, Wiepke Cahn, Vince D. Calhoun, Gianpiero L. Cavalleri, Benedicto Crespo-Facorro, Anders M. Dale, Gareth E. Davies, Norman Delanty, Chantal Depondt, Srdjan Djurovic, Wayne C. Drevets, Thomas Espeseth, Randy L. Gollub, Beng-Choon Ho, Wolfgang Hoffmann, Norbert Hosten, René S. Kahn, Stephanie Le Hellard, Andreas Meyer-Lindenberg, Bertram Müller-Myhsok, Matthias Nauck, Lars Nyberg, Massimo Pandolfo, Brenda W. J. H. Penninx, Joshua L. Roffman, Sanjay M. Sisodiya, Jordan W. Smoller, Hans van Bokhoven, Neeltje E. M. van Haren, Henry Völzke, Henrik Walter, Michael W. Weiner, Wei Wen, Tonya White, Ingrid Agartz, Ole A. Andreassen, John Blangero, Dorret I. Boomsma, Rachel M. Brouwer, Dara M. Cannon, Mark R. Cookson, Eco J. C. de Geus, Ian J. Deary, Gary Donohoe, Guillén Fernández, Simon E. Fisher, Clyde Francks, David C. Glahn, Hans J. Grabe, Oliver Gruber, John Hardy, Ryota Hashimoto, Hilleke E. Hulshoff Pol, Erik G. Jönsson, Iwona Kloszewska, Simon Lovestone, Venkata S. Mattay, Patrizia Mecocci, Colm McDonald, Andrew M. McIntosh, Roel A. Ophoff, Tomas Paus, Zdenka Pausova, Mina Ryten, Perminder S. Sachdev, Andrew J. Saykin, Andy Simmons, Andrew Singleton, Hilkka Soininen, Joanna M. Wardlaw, Michael E. Weale, Daniel R. Weinberger, Hieab H. H. Adams, Lenore J. Launer, Stephan Seiler, Reinhold Schmidt, Ganesh Chauhan, Claudia L. Satizabal, James T. Becker, Lisa Yanek, Sven J. van der Lee, Maritza Ebling, Bruce Fischl, W. T. Longstreth Jr, Douglas Greve, Helena Schmidt, Paul Nyquist, Louis N. Vinke, Cornelia M. van Duijn, Luting Xue, Bernard Mazoyer, Joshua C. Bis, Vilmundur Gudnason, Sudha Seshadri, M. Arfan Ikram, The Alzheimer?s Disease Neuroimaging Initiative, The CHARGE Consortium, EPIGEN, IMAGEN, SYS, Nicholas G. Martin, Margaret J. Wright, Gunter Schumann, Barbara Franke, Paul M. Thompson & Sarah E. MedlandNature, vol. 520(7546), pp. 224-229. 2015. Journal Paper


\end{backmatter}
\end{document}
